\input{preamble.tex}

\newcommand\sectio[1]{%
  \section*{#1}%
  \addcontentsline{toc}{section}{#1}}

\renewcommand\contentsname{Sommaire}

\renewcommand{\thesection}{\arabic{section}}
%\renewcommand{\thesubsection}{\arabic{section}}
\newcommand\familyname{\textsc}
\addto\captionsfrench{\renewcommand{\figurename}{Image}}

\newtoggle{annexes}
\toggletrue{annexes}

\input{first_page.tex}

\title{\textbf{Rapport de stage}}
\author{Marion \familyname{Ly} \\ Thibaut \familyname{Smith}}
\date{2013 - 2014}

\begin{document}
	\section{configuration des acteurs}
	
	Chaque acteur possède une même configuration :
	
	\begin{tabular}{|c|c|c|c|}
	\hline 
	Configuration/ Acteur & convoyeur & aiguilleurs & rfid  \\ 
	\hline 
	module & actorconveyor & actorrailway & actorrfid \\
	\hline 
	id & random & random & random \\
	\hline 
	opt & entrée/sortie & entrée/sortie/regle de propriorité & •  \\ 
	\hline 
	state & on/off & {[in],[out]} (son état physique) & • \\
	\hline 
	worktime & ? & ? & 0\\ 
	\hline 
	listdata & produit transféré & {produit tranféré, état aiguilleur} & produit détecté  \\ 
	\hline 
	\end{tabular} 
	
	
	\begin{tabular}{|c|c|c|}
	\hline
	Configuration/ Acteur & workstation & produits \\
	\hline
	module & actorworkstation & actorproduct \\ 
	\hline
	id & random & random \\ 
	\hline
	opt & entrée/sortie & •\\
	\hline
	state & on/off/default & etat pysique (raw,Q1,Q2,Q3) \\ 
	\hline
	worktime & ? & ?  \\
	\hline
	list\_data & {produit tranformé, qualité} & emplacement traversé\\
	\hline
	\end{tabular}
	
	\section{Interface : actor_contract}

	 une fonction create qui permet de créer l'acteur avec un id aléatoire pour l'instant et des configurations par défaut différentes pour chaque acteur
	
	 une fonction answer qui prend 2 arguments, le protocole utilisé dans les différents acteurs est 
	\begin{itemize}
	
	 \item pour l'appel à la fonction les arguments : {Config,{actor\_product, Config\_prod, type de 				question}}
	\item pour la réponse : {Config,{actor\_product, NewConfig\_prod, info}, destination}

	\end{itemize}
	 Le produit étant le seul acteur qui est un flux physique, il ne répond pas au protocole. 
	 L'interface comprend également différentes fonctions implémentées pour accéder, modifier les données des configurations.
	 
	 
	 \section{Lecteur RFID}
	 Lorsque le lecteur reçoit une demande du type 
	 
	 \section{Le produit}
	 
	 Comme dit précédemment, le produit est un flux physique donc il ne répond pas à un protocole pour ce qui est de ses réponses au superviseur. 
	 
\end{document}