\documentclass[12pt,a4paper]{article}
\usepackage[utf8]{inputenc}
\usepackage[french]{babel}
\usepackage[T1]{fontenc}
\usepackage{amsmath}
\usepackage{amsfonts}
\usepackage{amssymb}
\usepackage{pdflscape}
\usepackage{tabularx}
\usepackage[left=2cm,right=2cm,top=2cm,bottom=2cm]{geometry}
\usepackage{tikz}
\usetikzlibrary{arrows}

\setlength{\parskip}{1.1em}

\renewcommand\contentsname{Sommaire}

\renewcommand{\thesection}{\arabic{section}}
%\renewcommand{\thesubsection}{\arabic{section}}
\newcommand\familyname{\textsc}

\title{\textbf{Modélisation des Acteurs}}
\author{Marion \familyname{Ly} \\ Thibaut \familyname{Smith}}
\date{2013 - 2014}

\begin{document}

	\maketitle

	\tableofcontents

	\pagebreak

	\section{Interface pour les acteurs}

	Le module \textbf{actor\_contract} fournit à la fois la spécification d'un 
	acteur (les fonctions que l'acteur doit implémenter) et une liste de 
	fonctions exportées permettant de simplifier les actions les plus basiques 
	tels que création/recherche d'information dans les métadonnées d'un acteur.

	Voici une liste des fonctions que doivent implémenter chaque acteur pour 
	assurer une compatibilité avec le système de contrôle. 


	\subsection{Fonction create/0}
		Cette fonction renvoi un acteur avec des valeurs mises par défaut ou
		générés aléatoirement, tels que le nom de l'acteur (en général le nom
		du module), l'identifiant de l'acteur créé (unique).

		Elle ne prend aucun argument.

	\subsection{Fonction answer/2}
		Cette fonction permet de \textbf{caractériser} le comportement du 
		module en fonction d'une entrée précise (qui peut être un produit, mais
		pas nécessairement). C'est cette fonction qui doit modifier/lire le 
		produit si il y a lieu.

		Elle prend deux arguments en entrée:

		\begin{enumerate}
			\item La configuration de l'acteur
			\item La requête demandé qui s'exprime, pour toutes les 
			\textit{requêtes}, par le protocole suivant :

				\{ConfigurationActeurA, 
				\{NomActeurB, ConfigurationActeurB\}\}
				ActeurA correspond à l'acteur auquel on envoie la requête, et 
				ActeurB correspond au flux physique, dans notre cas le produit.
		\end{enumerate}

	 	Cette fonction représente aussi le temps de travail via une pause
	 	forcée (en mode temps-réel), c'est pourquoi à chaque fois que cette 
	 	fonction est appelée,
	 	elle est encapsulée dans un noeud extérieur qui prévient l'acteur 
	 	quand le travail est terminé. Cet aspect nous permet de contrôler 
	 	plusieurs requêtes en parallèle.

	 	La réponse est de la forme suivante :

	 	\{NouvelleConfigurationActeurA, 
	 	\{NomActeurB, ConfigurationActeurB, Information\},
	 	Destinaton\}

	 	\begin{enumerate}
	 		\item On retrouve les mêmes informations que tout à l'heure, mais 
	 			légèrement modifiées : la nouvelle configuration de l'acteur A 
	 			se retrouve en première position.
	 		\item le nom et la nouvelle configuration de l'acteur B en deuxième 
	 			position, et une information s'est rajoutée. Cette information 
	 			peut représenter l'information détectée par l'acteur (exemple : 
	 			le lecteur RFID met l'identifiant du produit ici), ou juste le 
	 			nom de l'action qui a été exécuté sur l'acteur B.
	 		\item Enfin vient la destination de la requête, qui est soit dirigée 
	 			à l'acteur suivant, soit au superviseur. (Cette valeur est 
	 			beaucoup utilisée par l'aiguilleur.)
	 	\end{enumerate}

	 	Cette fonction décrit donc à la fois les messages correspondant au
	 	flux physique et au flux logique. En fonction de la destination, 
	 	on peut faire la différence.

	 	La dernière tâche de cette fonction est d'inscrire des informations 
	 	utiles à la fois dans la base de données du produit et de l'acteur
	 	actuel.

		L'interface \textbf{actor\_contract} comprend également différentes 
		fonctions implémentées pour accéder et modifier les données des 
		configurations ainsi que des réponses types pouvant être utilisées 
		par chaque acteur. Il est cependant toujours possible d'annuler ce 
		comportement via programmation.
	 
	\begin{landscape}
		\begin{center}
			\begin{tabularx}{1.55\textwidth}{|X|X|X|X|X|X|X|}
				\hline
				Configuration & 
				Module & 
				Id & 
				opt (options) & 
				State & 
				Worktime & 
				list\_data \\
				\hline
				Convoyeur & 
				actor\_conveyor & 
				Manuel/Hasard & 
				sortie & 
				on/off & 
				Temps de transport & 
				Produits transférés \\
				\hline
				Aiguilleur & 
				actor\_railway & 
				Manuel/Hasard & 
				entrée/sortie/règle de priorité & 
				{in, out} (son état physique) & 
				Temps d'aiguillage & 
				Produits transférés, état aiguilleur \\
				\hline
				RFID & 
				actor\_rfid & 
				Manuel/Hasard & 
				entrée & 
				on/off & 
				Temps de détection & 
				Produits détectés\\
				\hline
				Station de travail & 
				actor\_workstation & 
				Manuel/Hasard & 
				entrée/sortie & 
				on/work/off & 
				Temps de fabrication & 
				Liste des produits crées, avec la qualité \\
				\hline
				Produit & 
				actor\_product & 
				Manuel/Hasard & 
				Qualité demandée & 
				raw/Q1/Q2/Q3 & 
				Aucun & 
				Emplacements parcourus \\
				\hline
				File d'attente basique & 
				actor\_basic\_queue & 
				Manuel/Hasard & 
				Pas d'options & 
				on/processing/off & 
				Temps minimum avant qu'un produit soit disponible & 
				Liste de produits avec temps d'entrée et temps de sortie \\
				\hline
				Scanner &
				actor\_scanner &
				Manuel/Hasard &
				Pas d'options &
				on/off &
				Temps de détection de 2 secondes &
				Liste de produits détectés, avec la qualité \\
				\hline
				Station de travail d'assemblage & 
				actor\_workstation \_assembly & 
				Manuel/Hasard &
				Configuration à insérer dans le produit suivant &
				on/processing/off &
				Temps d'assemblage de 12 secondes &
				Liste de produits transformés \\
				\hline
				Station de travail finition &
				actor\_workstation \_finish &
				Manuel/Hasard &
				Pas d'options &
				on/processing/off &
				5 secondes &
				Liste des produits avec la qualité \\
				\hline
			\end{tabularx}
		\end{center}
	
	\end{landscape}


	\section{Objectifs du module container}
		Le module \textbf{container} permet de créer un noeud Erlang qui 
		attend les messages des autres acteurs. Les fonctions du container
		ne dépendent uniquement de la configuration d'un acteur. (C'est pourquoi
		il est important que les acteurs implémentent l'interface.)
		Dans toute la suite de cette section, quand on dira `l'acteur', 
		il faudra en fait comprendre `le module container lancé avec la 
		configuration de tel acteur'.

		Le container utilise des fonctions pour la gestion de son acteur. Voici 
		une liste des choses qu'il doit gérer :

		\begin{enumerate}
			\item Recevoir les messages des autres acteurs
			\item Recevoir et envoyer les messages au superviseur
			\item Les noeuds qui représentent le travail sur les produits
			\item Les envois de produits aux autres acteurs
		\end{enumerate}

		\subsection{Recevoir les messages des autres acteurs}
			Les acteurs s'envoient principalement des produits entre eux, le 
			seul flux physique que nous avons.
			
			Les acteurs ayant à chaque instant un comportement asynchrone, 
			ils peuvent gérer les autres situations qui peuvent arriver.

		\subsection{Recevoir et envoyer les messages au superviseur}
			L'acteur envoie constamment ses changements de configurations
			au superviseur. Pour l'acteur aiguilleur, il y a un dialogue
			qui lui permet d'orienter le produit vers la bonne sortie.

		\subsection{Les noeuds qui représentent le travail sur les produits}
			Dès qu'un acteur reçoit un produit, le travail supposé être fait
			par l'acteur est effectué par la fonction \textit{answer/2}. Comme
			le travail à effectuer est potentiellement long, il est lancé
			dans un autre noeud en parallèle.

			Une fois que ce noeud a terminé, l'acteur est prévenu via les
			fonctions \textit{end\_of\_physical\_work/3} (\textbf{flux physique})
			et \textit{end\_of\_logical\_work/3} (\textbf{flux logiques}) et le produit
			est rajouté dans la liste des produits à envoyer.

		\subsection{Les envois de produits aux autres acteurs}
			Cependant, pour conserver une bonne gestion des produits vis-à-vis
			des blocages du aux capacités maximum des acteurs, on utilise le 
			protocole suivant quand un acteur souhaite envoyer un produit :

			\begin{enumerate}
				\item Le \textbf{premier acteur} envoie une notification de produit
					en attente au \textbf{second acteur}. Le produit est 
					enregistré dans le premier acteur, et il faut 
					\textbf{attendre la confirmation} du second acteur avant
					d'envoyer le produit.
					\footnote{Format : \{PidPremierActeur, awaiting\_product\}}
				
				\item Sur réception de cette notification, le second acteur 
					examine sa liste de produits qu'il a sous sa garde :
					les produits en cours de travail \textbf{et} les produits
					qui sont en attente d'être envoyé à l'acteur suivant.
					Deux issues possibles :
					\begin{itemize}
						\item Si le second acteur a de la place disponible, 
							il envoie une confirmation au premier acteur en
							utilisant le `PidPremierActeur' pour le contacter.
						\item Si il n'a pas de place, il n'envoie rien. Dès 
							que de la place se libère, il enverra une 
							confirmation au premier acteur.
					\end{itemize}

				\item Quand le premier acteur reçoit une confirmation,
					il envoie le \textbf{premier produit arrivé} (FIFO).
			\end{enumerate}

	\section{Acteurs actuellement implémentés}
	
		Voici des acteurs qui ont déjà été implémentés dans le but de pouvoir 
		mettre en place le scénario donné par nos encadrants.

		\subsection{Lecteur RFID}
		 	Lorsque le lecteur reçoit un produit, nous avons utilisé le temps
		 	de travail comme temps de détection (qui peut-être mit à zéro par
		 	défaut, s'il le faut) qui crée une pause.
		 	Puis le lecteur enregistre alors les données du produit dans 
		 	sa base et ses données à lui dans la base du produit. 
		 	Le lecteur envoie alors l'Id du produit qu'il a lu en réponse.
		 
		\subsection{Convoyeur}
			Lorsqu'un produit arrive au convoyeur, une pause correspondant au
			temps de travail de la configuration de l'acteur se passe, c'est
			le temps de transport du produit. Les données du produit sont 
			enregistrées dans la base du convoyeur, celles du convoyeur dans la 
			base du produit. Le produit est alors envoyé à l'acteur suivant.

		\subsection{Aiguilleur}
			À partir du moment, où l'aiguilleur reçoit le produit, sa réponse 
			dépendra des entrées sorties de celui-ci : 

			\begin{itemize}
				\item plusieurs entrées ou plusieurs sorties, un message est 
					envoyé au superviseur, contenant la description du conflit 
					et les données du produit et de l'aiguilleur pour que le 
					superviseur puisse prendre sa décision (un ordre de 
					priorité contenu dans les options de l'aiguilleur, et la 
					qualité demandé du produit peut l'aider).
				\item une entrée et une sortie, on ajoute le produit ainsi que 
					l'état de l'aiguilleur à la base de l'aiguilleur et on 
					envoie le produit à l'acteur suivant.
			\end{itemize}
			
			S'il y a eu conflit, l'aiguilleur recevra alors la décision du 
			superviseur sur l'aiguillage du produit. On pourra ajouter les 
			informations de la même façon qu'avec une entrée et une sortie. 
			L'aiguilleur changera d'état (entrée/sortie) et il y aura alors un 
			temps de travail. l'aiguilleur enverra alors ensuite le produit au 
			destinataire choisi par l'aiguilleur. Si la décision choisi est 
			identique qu'à l'état actuelle de l'aiguilleur celui-ci n'a donc pas 
			à changer d'état et il n'y a donc pas de temps de travail.

		\subsection{Station de travail}
			Les stations de travail peuvent recevoir un seul produit à la fois, 
			et permettent de modifier l'état du produit. On ne peut cependant
			pas le conna\^itre tant que le produit n'est pas arrivé à un acteur
			Scanner qui pourra confirmer l'état du produit réel.

			Cette acteur doit \^etre configuré avec l'option `workstation\_luck',
			correspondant à la qualité voulue des produits arrivant sur la
			station. Voici des exemples de valeurs :

			\begin{itemize}
				\item \textbf{\{'Q2', 33\}} : 33\% de chance d'accomplir un 
					objet de qualité Q2 (ceci est équivalent à une 
					équiprobabilité entre les trois qualités, Q1, Q2 et Q3).
				\item \textbf{\{'Q1', 98\}} : 98\% de chance d'accomplir un
					produit de qualité Q1 (configuration de la station de
					travail numéro deux dans le scénario final).
			\end{itemize}

		\subsection{File d'attente basique}
			Les files d'attentes sont juste avant les stations : elles 
			permettent de contrôler l'arrivée des objets selon la capacité
			de la station.

			Un temps minimum de transport est possible dans le cas où la file 
			d'attente est utilisé comme convoyeur : le temps représente le
			déplacement du produit. Cet élement est à zéro par défaut, soit le 
			fonctionnement de base.

			Plus tard, il sera possible d'implémenter des files spéciales à
			plusieurs entrées.
		 
		\subsection{Le produit}
			 Comme dit précédemment, le produit est un flux physique donc il 
			 n'a pas particulièrement besoin d'un protocole. On a donc désactivé
			 plusieurs type de requêtes qui permettent de le modifier. 
			 On lui demandera alors juste de répondre par ses données. 
			 Ce type de réponse pourra aussi être utilisé par les autres acteurs 
			 si on veut récupérer/modifier certaines de leurs données de 
			 manières simplifié.

			 Le produit est composé de la palette en bois et de la disposition
			 des tablettes. Il y a quatre emplacements possibles. De plus,
			 la qualité peut être augmenté une fois à 100\% gr\^ace à un acteur.

			 La qualité initiale du produit (palette en bois) est choisie de
			 manière équiprobable.

		\subsection{Le scanner}
			L'acteur Scanner permet de s'assurer de la qualité d'un produit.
			En effet, les stations de travail ne peuvent que travailler sur un
			produit et accomplir leur commande avec leur probabilité d'échouer.
			Le scanner se charge d'analyser la qualité du produit pour aider
			les superviseurs à diriger les acteurs.

			Note : il sera possible d'ajouter un pourcentage d'erreur à la 
			détection.

		\subsection{Station de travail d'assemblage}
			Cette acteur ajoute les tablettes dans le produit. On le configure
			avec l'option `order'. Voici des exemples :

			\begin{itemize}
				\item \textbf{\{1, 0, 1, 0\}}
				\item \textbf{\{1, 1, 1, 1\}}
				\item \textbf{\{0, 0, 0, 1\}}
			\end{itemize}

		\subsection{Station de travail pour la finition}
			Cette acteur se trouve dans la dernière zone du scénario final. Il
			permet d'augmenter la qualité d'un produit de un si la commande
			client le demande. De même que pour la station de travail 
			d'assemblage, la commande est à inscrire dans les options.

	\section{Diagramme d'état d'un container configuré avec un acteur}

		Voici le fonctionnement d'un acteur vis-à-vis des autres acteurs.
		\begin{center}
			\begin{tikzpicture}
			[->,
			 >=stealth',
			 shorten >=1pt,
			 auto,
			 node distance=8cm,
			 thick,
			 main node/.style={circle,draw,font=\sffamily\Large\bfseries}]

			  \node[main node] (1) {off};
			  \node[main node] (2) [right of=1] {waiting};
			  \node[main node] (3) [right of=2] {processing};
			  % \node[main node] (2) [below left of=1] {on};
			  % \node[main node] (3) [below right of=2] {processing};
			  % \node[main node] (4) [below right of=1] {4};

			  \path[every node/.style={font=\sffamily\small}]
			    (1) edge node {Allumage} (2)
			    (2) edge [loop above] node {Nouveau produit \textbf{et} place disponible} (2)
			        edge [bend right] node[below] {Nouveau produit \textbf{et} aucune place disponible} (3)
			    (3) edge node [above] {Traitement d'un produit terminé} (2);
			\end{tikzpicture}
		\end{center}

	\section{Tests unitaires}
		Sur demande de Arnould Guidat, nous avons programmé des tests unitaires
		pour le plus de fonctions possibles. Cela permet d'être sûr du 
		comportement des fonctions et de remarquer très rapidement les 
		régressions dans le code.

\end{document}
